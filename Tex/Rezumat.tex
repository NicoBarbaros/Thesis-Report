\section*{Rezumat}

Teza \textbf{Aplicație utilitară pentru kinetoterapia mâini} este prezentată de studentul Barbaroș Nicolae ca proiect de licență și a fost efectuată la Universitatea Tehnică din Moldova. Teza este scrisă in limba engleză și conține 66 pagini, 19 figuri, 22 de listări de cod și 16 referințe. Teza conține o listă de figuri, o listă de abrevieri, introducere, patru capitole, concluzie și o listă de referințe.



Teza are obiectivul de a cerceta solutiile existente pe piață în domeniul de reabilitate a unui pacient ce a avut un atac cerebral, o fractura la mână sau o interveție chirurgicală ce a adus la probleme de mobilitate a mâinilor și ulterior crearea unei aplicații ce va tinde să refacă durerile, leziunile de tendoane și dizabilitati neuromotorii ale mâinii.

Cu ajutorul tehnologiei noi Leap Motion, aplicația e capabilă să monitorizeze în realtime mâinile utilizatorilor. Utilizând datele oferite de Leap Motion despre activitățile mâinilor, le pot oferi pacienților un set de exerciții prin care iși vor putea începe etapa de reabilitate și un sistem de feedback constructiv prin care să-i înștiințeze pe utilizatori despre cât de corect fac exercițiile și numarul de exerciții efectuate. Dat fiindcă aplicația nu necesită manuși de monitorizare, tehnologii cu piese mari si multe camere de monitorizare, ci un dispozitiv mic, ușor, mai ieftin (dintre toate opțiunile pe piață) și respectiv care se conectează printr-un singur cablu USB, utilizatorul va fi capabil să țină permanent cu el dispozitivul și va putea exersa chiar și în avion, având doar aplicația instalată pe laptop.


Cele patru capitole ce compun raportul tezei sunt: problema, analiza domeniului și a soluției propuse în capitolul unu. Al doilea capitol este despre diagramele UML care descriu sistemul și beneficiile utilizatorului, urmat de capitolul trei cu tehnologiile utilizate și implementarea codului sursă, finalizând cu capitolul despre analiz economică a sistemului. Primul capitol descrie problema care a fost identificată și descrie soluția care a fost găsită. De asemenea este făcută o analiză a pieței locale și a soluțiilor existente. În capitolul doi sunt prezentate tehnologiile și cîteva instrucțiuni de bază despre cum putem să le utilizăm. De asemenea este descrisă implementarea codului sursă cu exemple de cod listate. În capitolul trei sunt prezentate diagramele UML ale sistemului și beneficiile utilizatorului pentru a folosi acest sistem. În ultimul capitol este facută o analiză economică, sunt prezentate cheltuielile, veniturile posibile și este analizată profitabilitatea proiectului. Acest document este destinat pentru cititori cu cunoștințe în domeniul technic, ingineri, studenți TI și programatori.
