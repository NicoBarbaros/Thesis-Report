\section*{Rezumat}

Teza \textbf{Modele matematice şi metode de eficientizare a conversiei energiilor regenerabile
 în baza efectelor aero-hidrodinamice}, prezentată de către Viorel Bostan pentru 
conferirea gradului ştiinţific de doctor habilitat în tehnică, a fost elaborată la Universitatea Tehnică a Moldovei, Chişinău, este scrisă în limba română şi conţine 342 pagini, 90 de figuri, 38 tabele, şi 250 de titluri bibliografice. Structura tezei include: introducerea, 6 capitole, concluzii şi anexe. Anexele conţin 145 de pagini cu 52 de figuri şi 48 tabele.
	
Teza este consacrată studiului fenomenelor aero-hidrodinamice în rotoarele turbinelor eoliene (TE) şi microhidrocentralelor de flux (MCHF) de mică putere ($P<20$ kW), cu aplicarea modelelor matematice de descriere a fizicii curgerii fluidelor şi a metodelor moderne de simulare numerică a turbulenţei din cadrul dinamicii fluidelor CFD.
	
Scopul lucrării constă în sporirea eficienţei conversiei şi a capacităţilor funcţionale ale turbinelor eoliene şi microhidrocentralelor de flux de mică putere.
	
Au fost identificate modelele şi metodele matematice moderne de descriere a curgerii turbulente a fluidului, specifică rotoarelor de mică putere, cu evidenţierea efectelor aero-hidrodinamice tranzitorii şi în vecinătatea palelor. A fost argumentate profilurile aero-hidrodinamice ale palelor eficiente din punct de vedere al randamentului conversiei energiei şi în baza lor au fost elaborate concepte originale de rotoare aero-hidrodinamice.
	
În baza modelelor CAD ale rotoarelor propuse: au fost efectuate simulări CFD complexe ale curgerii tranzitorii a fluidului prin rotoare şi în vecinătatea palelor, cu determinarea gradului de influenţă a parametrilor constructiv-cinematici asupra caracteristicilor de putere şi factorilor de performanţă aero-hidrodinamică a rotoarelor TE şi MHCF; a fost efectuată analiza fenomenului de curgere a fluidului în stratul limită şi identificate soluţii tehnice de control şi minimizare a impactului negativ al acestuia asupra eficienţei conversiei energiei.

În baza rezultatelor cercetărilor, au fost elaborate şi fabricate modele noi de TE şi MHCF pentru diverse aplicaţii, inclusiv conceptul TE cu rotor basculant şi orientare la direcţia curenţilor de aer cu windrose cu profil aerodinamic al palelor. Soluţiile tehnice elaborate au fost protejate cu 17 brevete de invenţie şi apreciate la saloanele internaţionale de inovaţii, cercetare şi transfer tehnologic cu 43 medalii de aur, 13 de argint şi 2 de bronz.

Cuvinte-cheie: modele matematice; simulare numerică CFD; strat limită; curgere turbulentă, rotor aero-hidrodinamic, turbină eoliană; microhidrocentrală.