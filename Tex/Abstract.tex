\section*{Abstract}


The \textbf{Utility Application for Hand Kinetotherapy} presented by student Barbaroș Nicolae as a Bachelor project and it was developed at the Technical University of Moldova. It is written in English and contains 66 pages, 19 figures, 22 listings and 16 references. The thesis consist of a list of figures, list of abbreviations, list of listings,introduction, four chapters, conclusions, and references list.


The thesis has the object of studying the existing solution in the market of a rehabilitation system of an patient that got a stroke, hand fracture or a surgery and followed hand pain or hand paralysis and eventually creation of an application that  will aim to treat patients with hand pain, tendons injuries and neuro disorders.


With the help of Leap Motion technology, the application is capable to track the users hands in realtime. Using the tracked data offered by the controller about hand activities, the user is provided with a set of well defined exercises for a great rehabilitation experience and a feedback sistem so that every time the user will do an exercise, he will be notified about the correctness of the did exercise and the counted done exercises.

Hence the application does not require the use of extra complicated objects that will track users hand and only a small, easy, cheaper device that is connected only by an USB cable to the laptop, the user will be capable of having the device by his side and even in an airplain he can connect the device to the laptop, open the application, and start exercising.


The four chapter which compose the report are: the problem, domain analysis and the purposed
solution chapter. The second one is the used technologies and implementation chapter, followed by
the UML description of the system and user experience chapter, concluding with the chapter four
about the economical analysis of the system. The first chapter describes the problem that was identified and defines the solution. It also has a deep analysis of the market and existing solutions.
In the chapter two are presented the UML diagrams of the system and is listed the benefits of the user when using this application. In chapter three are presented the used technologies and some basic instructions on how to use
them. Also there is described the implementation part with sample listings of code. In the last
chapter is made a economical analysis in which are shown the total expenses, possible incomes and
is analysed the profitability of the project. This document is for readers with technical background,
engineers, IT students and programmers
