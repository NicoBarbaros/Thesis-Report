\section*{Conclusions}
\phantomsection

Kyno project is the result of the thesis work. It is an application that gives to the user exercises for kinetotherapy rehabilitation. The uniqueness of the application is that the targeted users and the proposed solution is unusual for the moldavian market. There are still no tools available that does a similar thing on the local level. However, developing a software that doesn't yet exist on a market offers a lot of freedom, but at the same time is problematic because the needs of the potential customers are yet blurry. A joint work with a the Physiotherapist representative would have resulted into a more specific and useful product.

There are other rehab systems out there that use motion capture, but they often require sensor gloves or other proprietary hardware that take a lot of training and supervision to use, or they depend on rigging an entire room with expensive cameras or placing lots of sensors on the body. Thanks to Leap Motion, Kyno doesn’t require any extra hardware, cameras, or body sensors, which keeps the price affordable,. That low price point is extremely important.

Kyno is working to remove all the major barriers to physical rehabilitation by making a system that is fun, simple to use, and affordable. Kyno demonstrates the potential of NUI to make technology simpler and more effective—and the ability of Leap Motion technology to help high tech meet essential human needs.

Building the application required many multi-step planning of the infrastructure. The first problem encountered with choosing what gestures to be detected in the app. Due to the Leap Motion device tracking limitation some of the exercises can not be included because Leap Motion is not able to "see through the fingers" - for example, when one finger covers the other. Fingers right next to each other also pose a problem for the cameras and might not be recognized individually. That's why some exercises cannot be done.


During first stage of marketing it is really important to collect feedback about the application from the clients and then releasing new versions with the implemented feedback. Also if the marketing want's to cross the border and wants to grow on international levels, the more functionalities should be implemented in the sistem.

One of the functionality to be implemented is a system that is  monitoring the progress of patients from anywhere in the world. Patients can perform complex rehabilitation programs using entertaining therapies either in a Rehabilitation clinic as well as in their own homes.

Each session will be registered using Microsoft® Azure, a cloud-based platform that will allow patients to complete therapy sessions under the supervision of their specialist, whether at a medical centre or at home.

An important part for the future development of Kyno is to add mini artificial inteligent component that will take the user through an amazing experience on how to use the application, how to put the hands over Leap Motion controller, how to perform certain exercises. So basically the component will tell the user to press one button so the user will have to do that, then it will say to the user to put his hands over Leap Motion and again user will have to act as the component shows. It will be runned every time the application was launched for the first time on a system after that it will just ask the user if he want to go through a tutorial.

Another future implementation is adding more exercises, which means detecting more gestures, then gamify them to create a better user experience. Gamified content makes users to establish a special excited relation between them and the application.


As a generalization, working on this project was an extreme enjoyable challenge for me. Since  the goal was to  write a full documentation of it in the form of a report, to devide into time periods the process of development and work constanly on the application.


\clearpage