\section{ Implementation and Used Technologies}
\phantomsection

For a better understanding of the application, on how it works, how it behaves for the patient and in general for those who will use it, it is better and I would say important to describe the technologies that were used to build Kyno. In this chapter will be analysed the implementation of the features that presents the Kyno as a software, will be shown some code samples and analysed the solutions that were choosed to develop the application.

\subsection{System Requirements}
In order to run the application and to have a nice user experience, there is needed for the client to have 2 componenets.

\begin{itemize}
\item A computer, laptop or any system/device where you can install a desktop application

\item A Leap Motion tracking device with again the ability to install software on your device.
\end{itemize}

As we can see the requirements are not that simple and easy to get since the Leap Motion can be bought only online from USA and cost almost 100 euros. However the application will come with the Leap Motion controller in the set.

\subsection {Technologies Used}
\subsubsection {Unity}
\subsubsection {Leap Motion}
\subsubsection {C\# or JavaScript?}
\subsubsection {Leap Motion SDK}
\subsection {LitJson}
 \autoref{bootstrap}. This simple structure makes it easy to locate the files and edit them when it's needed.
\begin{lstlisting}[caption={Bootstrap folder structure},label={bootstrap}]
bootstrap/
    css/
        bootstrap.css
        bootstrap.css.map
        bootstrap.min.css
        bootstrap-theme.css
        bootstrap-theme.css.map
        bootstrap-theme.min.css
    js/
        bootstrap.js
        bootstrap.min.js
    fonts/
        ...
\end{lstlisting}
\clearpage